%!TEX TS-program = LuaLateX
\documentclass[12pt,ngerman]{dtk}
\usepackage[utf8]{inputenc}
\usepackage{hyperref}
\usepackage{showexpl,ifthen}
\usepackage{xspace,listings,xcolor}
\usepackage{babel}
\usepackage{microtype}
\usepackage[utf8]{inputenc}
\usepackage{graphicx}
\usepackage{csquotes}
\usepackage{url}
\usepackage{calc,tikz}
\usetikzlibrary{positioning,calc}

\definecolor{hellgelb}{rgb}{1,1,0.8}
\definecolor{colKeys}{rgb}{0,0,1}
\definecolor{colIdentifier}{rgb}{0,0,0}
\definecolor{colComments}{rgb}{1,0,0}
\definecolor{colString}{rgb}{0,0.5,0}

\lstset{language={[LaTeX]TeX},
    float=hbp,%
    basicstyle=\ttfamily\small, %
    identifierstyle=\color{colIdentifier}, %
    keywordstyle=\color{colKeys}, %
    stringstyle=\color{colString}, %
    commentstyle=\color{colComments}, %
    columns=flexible, %
    tabsize=2, %
	morekeywords={draw,node,grid,coordinate},
    extendedchars=true, %
    showspaces=false, %
    showstringspaces=false, %
    backgroundcolor=\color{hellgelb}, %
    breakautoindent=true, %
    captionpos=b%
}



\lstset{
  literate={ö}{{\"o}}1
           {ä}{{\"a}}1
           {ü}{{\"u}}1
}

\title{3D-Objekte mit TikZ -- Eine \TeX nische Reise} 
\Author{Uwe}{Ziegenhagen}{Köln}

%\markboth{Python\TeX}{Python\TeX}

\begin{document}
\maketitle

\begin{abstract}
Ausgehend von eine Frage im \LaTeX-Forum von Reddit betrachte ich in diesem Artikel einige Möglichkeiten, 3D-Objekte \enquote{auf Papier} respektive ins PDF zu bringen.
\end{abstract}

\section{Der erste Versuch}

Es war im März diesen Jahres, als ein Nutzer im \LaTeX-Forum von Reddit (\url{https://www.reddit.com/r/LaTeX/}) fragte, wie man Grafiken analog zur der in Abbildung \ref{fig:template} in \LaTeX\ erstellen könnte.  Ich nahm dies als willkommene Gelegenheit, mal wieder ein wenig zu \enquote{T(r)ikZen}. TikZ bietet verschiedene Möglichkeiten für die Erstellung,  die wir uns im folgenden einmal anschauen werden.

\begin{figure}[h]
\begin{center}
\includegraphics[width=0.75\textwidth]{template}
\caption{Die gewünschte Grafik, hier behelfsweise mit Affinity Designer erzeugt}\label{fig:template}
\end{center}
\end{figure}


\section{Verbesserungen}

\section{Der Sprung in die dritte Dimension}

\section{Spielereien}


\end{document}

%https://www.reddit.com/r/LaTeX/comments/1bb99v5/how_do_i_make_these_plots/
